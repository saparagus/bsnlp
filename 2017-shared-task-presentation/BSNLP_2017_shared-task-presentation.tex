\documentclass{beamer}

\newcommand{\lang}{\stackrel{\rightarrow}{\mathcal{L}}}

%\usepackage{multimedia}

%\usepackage{pgf,pgfarrows,pgfnodes,pgfautomata,pgfheaps,pgfshade}
\usepackage{epic}
\usepackage{eepic}
\usepackage{epsf}
\usepackage{times}
\usepackage{boxedminipage}
%\usepackage[polish]{babel}
\usepackage[utf8]{inputenc}
\usepackage[T1]{fontenc}
\usepackage{times}
\usepackage{latexsym}
\usepackage{amsmath}
\usepackage{color}
\usepackage{multirow}
\usepackage{graphicx}
\usepackage{booktabs}
\usepackage{fixltx2e}

\usepackage[russian,english]{babel}
\newcommand\textcyr[1]{{\fontencoding{T2A}\selectfont #1}}

%\usepackage{accents}
%\usepackage{hyperref}

%\usepackage{polski}

\newcommand{\comment}[1]{}

\setbeamertemplate{background canvas}[vertical shading][]

%\setbeamertemplate{itemize item}[ball]
\usetheme{Warsaw}
%\usebackgroundtemplate{\includegraphics[width=\paperwidth, height=\paperheight]{FrontexTemplate.jpg}}
\usefonttheme[onlysmall]{structurebold}
\usecolortheme[named=blue]{structure}


\definecolor{darkblue}{RGB}{0,51,153}
\definecolor{darkgreen}{RGB}{0,102,0}
\definecolor{darkred}{RGB}{204,0,0}
  

\setbeamertemplate{frametitle}
{
\vspace{0.3cm}\hspace{-0.5cm}
\color{darkblue}
\textbf{\insertframetitle}
\par
}

\setbeamertemplate{navigation symbols}{}

\definecolor{gold}{rgb}{0.85,.66,0}

\bibliographystyle{alpha}



\title[Cross-lingual NER Challenge]{\textbf{The First Cross-Lingual Challenge on Recognition, Normalization,
and Matching of Named Entities in Slavic Languages}}
\author[J. Piskorski et. al]{Jakub Piskorski, Lidia Pivovarova, Jan Šnajder,\\ Josef Steinberger, Roman Yangarber\\}   
%\institute{{Put logos here}
%}

\titlegraphic{\includegraphics[scale=0.3]{institution-logos.png}}


\date{{\color{white} 4 April 2017, Valencia, Spain}}


\begin{document}

\maketitle

%\usebackgroundtemplate{\includegraphics[width=\paperwidth, height=\paperheight]{AutomatedEventExtraction_firstPage.jpg}}
%\begin{frame}[plain]
%
%\end{frame}

%\usebackgroundtemplate{\includegraphics[width=\paperwidth, height=\paperheight]{FrontexBlue.jpg}}

%****************
% SLIDE 
%****************

\begin{frame}
 \frametitle{Outline}

\begin{itemize}
\item Motivation
\item Tasks
\item Trial and Test Datasets
\item Baseline System: LexiFlexi
\item Evaluation Methodology
\item Evaluation Results
\item Way Forward
\end{itemize}

%\sound[sound[encoding=Signed]{HERE}{test-sound.aiff}

\end{frame}

\begin{frame}
 \frametitle{Motivation}

\end{frame}

\begin{frame}
 \frametitle{Tasks}

\begin{itemize}

\item \textbf{Named Entity Mention Detection and Classification}

\begin{itemize}

\item \textbf{ORG} (ex. {\color{blue}\textit{Citi Handlowy w Poznaniu}} - PL)

\vspace{0.2cm}

\item \textbf{PER} (ex. {\color{blue}\textit{Władimir Putin}} - PL, {\color{blue}\textit{Ukrajinci}} - SI)

\vspace{0.2cm}

\item \textbf{LOC} toponyms, GPEs, facilities, (ex. {\color{blue}\textit{Rusko}} - CS, {\color{blue}\textit{Európska únia}} - SK, {\color{blue}\textit{Zagrebački Glavni kolodvor}} - HR)

\vspace{0.2cm}

\item \textbf{MISC} (ex. {\color{blue}\textit{Motorola Moto X}}, {\color{blue}\textit{Święta Bożego Narodzenia}} - PL)

\vspace{0.2cm}

\item no extraction of positional information

\vspace{0.2cm}

\item recognition of timex, numex and identifiers and nested NEs \textbf{not part of the task}

\end{itemize}

\end{itemize}

\end{frame}

\begin{frame}[fragile]
 \frametitle{Tasks}

\begin{itemize}

\item \textbf{Name Normalization}

\begin{table}
  \begin{center}
    \begin{footnotesize}
      % \begin{tabular}{|l|c|c|c|c|c|c|}
\resizebox{1.02\linewidth}{!}{
      \begin{tabular}{lll}
        \toprule 
        \comment{Lan} & {Genitive} & {Nominative (``base'')}\\
        \midrule
        hr & {\color{blue}Europske komisije} & {\color{blue}Europska komisija} \\
        cz & {\color{blue}Komisji Europejskiej} & {\color{blue}Komisja Europejska} \\
        pl & {\color{blue}Evropskou komisí} & {\color{blue}Evropská komise} \\
       ru & {\color{blue}\textcyr{Европейской комиссией}} & {\color{blue}\textcyr{Европейская комиссия}} \\
        sl & {\color{blue}Evropske komisije} & {\color{blue}Evropska komisija} \\
        sk & {\color{blue}Európskej komisie} & {\color{blue}Európska komisia} \\
        ua & {\color{blue}\textcyr{Європейської Комісії}} & {\color{blue}\textcyr{Європейська Комісія}} \\
        \bottomrule
      \end{tabular}
}
    \end{footnotesize}
  \end{center}
\end{table}

\item \textbf{Entity Matching}



\end{itemize}

\end{frame}

\begin{frame}[fragile]
 \frametitle{Trial and Test Datasets}

\begin{itemize}

\item Trial Dataset: 

\begin{itemize}

\item a dataset of \textbf{187 docs} related to {\color{red}Beata Szydło}, the
current prime minister of Poland, 

\item a dataset of \textbf{186 docs} related to {\color{red}ISIS}

\end{itemize}

\item Test Datasets:

\begin{itemize}

\item a dataset of \textbf{177 docs} related to {\color{red}Donald Trump}, 

\item a dataset of \textbf{203 docs} related to {\color{red}European Commission}

\end{itemize}

\item Languages: Czech, Croatian, Polish, Slovak, Slovenian, Russian and Ukrainian

\end{itemize}

\end{frame}

\begin{frame}[fragile]
 \frametitle{Test Datasets}

\begin{table}
  \begin{center}
    \begin{footnotesize}
      % \begin{tabular}{|l|c|c|c|c|c|c|}
      \begin{tabular}{lcccc}
        \toprule 
        & \multicolumn{2}{c}{\textbf{{\sc Trump}}} & \multicolumn{2}{c}{\textbf{{\sc ECommission}}} \\
        \cmidrule(lr){2-3}
        \cmidrule(lr){4-5}
        Language &  \#\,docs & \#\,ment & \#\,docs & \#\,ment \\
        \midrule
        Croatian & 25 & 525 & 25 & 436 \\
        Czech & 25 & 479  & 25 & 417 \\
        Polish & 25 & 692  & 24 & 466 \\
        Russian & 26 & 331  & 24 & 385 \\
        Slovak  & 24 & 453  & 25 & 374 \\
        Slovene & 24 & 474  & 26 & 434 \\
        Ukrainian & 28 & 337  & 54 & 1078 \\
        \midrule
        Total & 177 & 3291  & 203 & 3588 \\

        \bottomrule
      \end{tabular}
    \end{footnotesize}
  \end{center}
	\caption{Quantitative data about the test datasets.}
\end{table}

\end{frame}

\begin{frame}[fragile]
 \frametitle{Test Datasets}

%%%%%%%%%%%%%%%%%%%%%%%%%%%%%%%%%%%%%%%%%%%%%%%%%%%%%%%%%%%%%%%%%%%%%%%%%%%%%%
\begin{table}
  \begin{center}
    \begin{footnotesize}
      % \begin{tabular}{|l|c|c|c|c|c|c|}
      \begin{tabular}{lcc}
        \toprule 
        Entity type & {\textbf{{\sc Trump}}} & {\textbf{{\sc ECommission}}} \\
        \midrule
	\textsc{Per} & 48.4\% & 11.9\% \\
	\textsc{Loc} & 26.9\% & 29.1\% \\
	\textsc{Org} & 18.3\% & 48.4\% \\
	\textsc{Misc} & \phantom{0}6.4\% & \phantom{0}9.6\% \\
        \bottomrule
      \end{tabular}
    \end{footnotesize}
  \end{center}
  \caption{Breakdown of the annotations according to the entity type.}
\end{table}

Inflected forms: in {\sc Trump} dataset min {\color{red}37.5\%} (Slovak) and max {\color{red}57.5\%} (Croatian)

\end{frame}

\begin{frame}[fragile]
 \frametitle{Test Datasets: Preparation}

\begin{itemize}

\item pose a search query to Google in each of the target languages

\item extract max. 100 links and remove duplicates

\item download documents, parse HTML and convert to pure text

\item remove documents with obvious HTML parser failure

\item select for each language/topic circa 25 documents for annotation (1 person per language)

\item 2 persons aligned the cross-language IDs

\end{itemize}

\end{frame}

\begin{frame}[fragile]
 \frametitle{Baseline system: Lexi Flexi}

\textbf{Basic idea:} exploit existing lexico-semantic resources publicly available

\begin{enumerate}

\item match names from {\color{red}JRC-Names} database (4,05 mln entries) + exploit the cross-lingual entity IDs,

\item match names from a collection of \textbf{multi-word named entities} semi-automatically derrived from {\color{red}{\sc Babelnet}} (6,82 mln entries) in unconsumed text,

\item match toponyms from the {\color{red}{\sc GeoNames}} gazetteer (1,36 mln) in unconsumed part of the texts + exploit cross-lingual IDs,

\item apply language-independent heuristics to match variants of mentions of entities recognised in the previous steps

\end{enumerate}

\end{frame}

\begin{frame}
 \frametitle{Evaluation Methodology}

\end{frame}


\begin{frame}
 \frametitle{Participant Systems}

\end{frame}

\begin{frame}
 \frametitle{Evaluation Results}

\end{frame}


\begin{frame}
 \frametitle{Way Forward}

\begin{itemize}

\item provision of additional test datasets (of similar nature)

\item extend the set of the languages covered (inclusion of Baltic languages?)

\item refining the NE annotation guidelines

\item making the evaluation software publicly available

\end{itemize}

\end{frame}

\end{document}