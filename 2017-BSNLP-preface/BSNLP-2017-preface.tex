\section{Preface}

This volume contains the papers presented at BSNLP-2017: the Sixth
Workshop on Balto-Slavic Natural Language Processing.  The Workshop is
organized by SIGSLAV---Special Interest Group on NLP in Slavic Languages
of the Association for Computational Linguistics.

The Workshops have been convening for over a decade, with a clear vision
and purpose.  On one hand, the languages from the Balto-Slavic group play
an important role due to their widespread use and diverse cultural
heritage.  These languages are spoken by about one third of all speakers
of the official languages of the European Union, and by over 400 million
speakers worldwide.  The political and economic developments in Central
and Eastern Europe place societies where Balto-Slavic languages are
spoken at the center of rapid technological advancement and the growing
European consumer markets.

On the other hand, research on theoretical and applied NLP in some of
these languages still lags behind the ``major'' languages, such as
English and other West European languages.  In comparison to English,
which has dominated the digital world since the advent of the Internet,
many of these languages still lack resources, processing tools and
applications---especially those with smaller speaker bases.

The Balto-Slavic languages pose a wealth of fascinating scientific
challenges.  The linguistic phenomena specific to the Balto-Slavic
languages---complex morphology and free word order---present non-trivial
problems for construction of NLP tools, and require rich morphological
and syntactic resources.  With a view toward that, in his invited talk on
``Pan-Slavic NLP,'' Serge Sharoff discusses his ambitious work on
Language Adaptation: methods for efficient adaptation of tools and
resources among closely related languages, such as those in the Slavic
group.

The BSNLP Workshops aim to bring together academic researchers and
industry specialists in NLP for Balto-Slavic languages.  We aim to
stimulate research and to foster the creation and dissemination of tools
and resources.  The Workshop serves as a forum for exchange of ideas and
experience and for discussing shared problems.  One fascinating aspect of
this group of languages is their structural similarity, as well as an
easily recognizable lexical and inflectional inventory spanning the
entire group, which---despite the lack of mutual
intelligibility---creates a special environment in which researchers can
fully appreciate the shared problems and solutions.

As a result of discussions at the previous BSNLP Workshops, to help
catalyze collaboration, this year we have organized the first SIGSLAV
Challenge: a shared task on multilingual named entity recognition.  We
have built a dataset, which allows systems to be evaluated on recognizing
mentions of named entities in Web documents, their
normalization/lemmatization, and cross-lingual matching.  The Challenge
initially covers seven Slavic languages, and it is intended as a first
version of an evaluation standard to be expanded in the future.

We received 24 regular submissions, 14 of which were accepted for
presentation.

The papers cover a wide range of topics.  Two papers relate to lexical
semantics, four to development of linguistic resources, and four to
information filtering, information retrieval, and information extraction.
Four papers cover topics related to processing of non-standard language
or user-generated content.
One paper describes the Challenge.

Additionally, 11 teams from 10 countries expressed interest in
participating in the Named Entity Challenge, of which two teams have
submitted results and system descriptions to date, and whose work is
discussed during the session dedicated specifically to the Challenge.

Overall, this workshop's presentations cover at least 10 Balto-Slavic
languages: Croatian, Lithuanian, Polish, Russian, Rusyn, Slovene, Serbian
(via the regular Workshop papers), and additionally Czech, Slovak and
Ukrainian (via the Shared Task Challenge).

This Workshop continues the proud tradition established by the earlier
BSNLP Workshops, which were held in conjunction with:

\begin{enumerate}
\item ACL 2007 Conference in Prague, Czech Republic,
\item IIS 2009: Intelligent Information Systems, in Kraków, Poland,
\item TSD 2011: 14th International Conference on Text, Speech and Dialogue in Plzen, Czech Republic,
\item ACL 2013 Conference in Sofia, Bulgaria,
\item RANLP 2015 Conference in Hissar, Bulgaria.
\end{enumerate}

We sincerely hope that this work will help further stimulate further
growth of our rich and exciting field.

BSNLP Organizers:

Jakub Piskorski (Joint Research Centre)
Lidia Pivovarova (University of Helsinki)
Jan Šnajder (Univeristy of Zagreb)
Roman Yangarber (University of Helsinki)
