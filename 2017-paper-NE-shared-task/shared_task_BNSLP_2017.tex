%
% File eacl2017.tex
%
%% Based on the style files for ACL-2016
%% Based on the style files for ACL-2015, with some improvements
%%  taken from the NAACL-2016 style
%% Based on the style files for ACL-2014, which were, in turn,
%% Based on the style files for ACL-2013, which were, in turn,
%% Based on the style files for ACL-2012, which were, in turn,
%% based on the style files for ACL-2011, which were, in turn, 
%% based on the style files for ACL-2010, which were, in turn, 
%% based on the style files for ACL-IJCNLP-2009, which were, in turn,
%% based on the style files for EACL-2009 and IJCNLP-2008...

%% Based on the style files for EACL 2006 by 
%%e.agirre@ehu.es or Sergi.Balari@uab.es
%% and that of ACL 08 by Joakim Nivre and Noah Smith

\documentclass[11pt]{article}
\usepackage{eacl2017}
\usepackage{times}
\usepackage{url}
\usepackage{latexsym}
\usepackage{amsmath}
%\usepackage{polski}
\usepackage[utf8]{inputenc}
\usepackage[T1]{fontenc}
\usepackage{booktabs}
\usepackage{color}
\usepackage{multirow}

%\usepackage{todonotes}%remove this when all todos are gone
\usepackage{todonotes}

\newcommand{\alert}[1]{\textcolor{red}{#1}}

\newcommand{\comment}[1]{}
\newcommand{\COMMENT}[1]{\comment{#1}}

\newcommand{\nocomment}[1]{{#1}}
\newcommand{\SHORTEN}[1]{}
\newcommand{\textexample}[1]{{\em #1}}
\newcommand{\dontsubmit}[1]{{\color{blue} #1}} %Change this to {} once ready for submission
%\newcommand{\dontsubmit}[1]{} %Change this to {} once ready for submission
\newcommand{\URGENT}[1]{\dontsubmit{\bf \color{red} #1}} %{{\bf \color{red} #1}}% % replace by {} when done




\eaclfinalcopy % Uncomment this line for the final submission
%\def\eaclpaperid{***} %  Enter the acl Paper ID here

%\setlength\titlebox{5cm}
% You can expand the titlebox if you need extra space
% to show all the authors. Please do not make the titlebox
% smaller than 5cm (the original size); we will check this
% in the camera-ready version and ask you to change it back.

\newcommand\BibTeX{B{\sc ib}\TeX}


% Shared Task on Multilingual Named Entity Recognition, Normalisation and
% Cross-language Matching for Slavic Languages

\title{The First Cross-Lingual Challenge on Recognition,
  Normalization,  
  \\ and Matching of Named Entities in Slavic Languages}


  \author{Jakub Piskorski\textsuperscript{1},
  Lidia Pivovarova\textsuperscript{2},
  Jan Šnajder\textsuperscript{3},
  Josef Steinberger\textsuperscript{4},
  Roman Yangarber\textsuperscript{2} \\
  %
  \textsuperscript{1}Joint Research Centre,
  \comment{Via Enrico Fermi 2749, 21027}Ispra\comment{ (VA)}, Italy,
  {\small \tt jakub.piskorski@jrc.ec.europa.eu} \\
  %
  \textsuperscript{2}University of Helsinki, Finland, {\small \tt
  \{roman.yangarber,lidia.pivovara\}@cs.helsinki.fi}\\
  %
    \textsuperscript{2}University of Zagreb, Croatia, {\small \tt
    jan.snajder@fer.hr} \\
  %
  \textsuperscript{3}University of West Bohemia, Czech Republic, 
  {\small \tt first.last@kiv.zcu.cz} 
  %
  }

\comment{author 
  Jakub Piskorski \\
  Joint Research Centre \\
  Via Enrico Fermi 2749 \\
  21027 Ispra (VA), Italy \\
  {\small \tt first.last@jrc.ec.europa.eu} \\\And
  Lidia Pivovarova \\
  Affiliation / Address line 1 \\
  Affiliation / Address line 2 \\
  Affiliation / Address line 3 \\
  {\tt email@domain} \\\And 
	Jan Šnajder \\
  Affiliation / Address line 1 \\
  Affiliation / Address line 2 \\
  Affiliation / Address line 3 \\
  {\tt email@domain} \\\And	
	Josef Steinberger \\
  Affiliation / Address line 1 \\
  Affiliation / Address line 2 \\
  Affiliation / Address line 3 \\
  {\tt email@domain} \\\And
	Roman Yangarber \\
  Affiliation / Address line 1 \\
  Affiliation / Address line 2 \\
  Affiliation / Address line 3 \\
  {\tt email@domain} 
	\\}

\date{}

\begin{document}
\maketitle
\begin{abstract}

This paper describes the outcomes of the first challenge on multilingual named entity recognition 
that aimed at recognizing mentions of named entities in web documents in Slavic languages, 
their normalization/lemmatization, and cross-language matching. It was organised in 
the context of the 6th Balto-Slavic Natural Language Processing Workshop, co-located with the 
EACL 2017 conference. Although eleven teams signed up for the evaluation, due to the complexity
of the task(s) and short time available for elaborating a solution, only two teams submitted results
on time. The reported evaluation figures reflect the relatively higher level of complexity of 
named entity-related tasks in the context of processing texts in Slavic languages. Since the 
duration of the challenge goes beyond the date of the publication of this paper and updated
picture of the participating systems and their corresponding performance can be found on
the web page of the challenge.

\end{abstract}

\section{Introduction}
\label{sec:intro}

Due to rich inflection, derivation, free word order, and other phenomena
exhibited by Slavic languages, the detection of names and their
lemmatization poses a challenging
task~\cite{Przepiorkowski:2007:SIE:1567545.1567547,journals/ir/PiskorskiWS09}.
Fostering research and development on this problem---and the closely
related problem of entity linking---is of paramount importance for
enabling effective multilingual and cross-lingual information access in
these languages.

This paper describes the outcomes of the shared task on multilingual
named entity recognition {(NER)} that aimed at recognizing mentions
of named entities in web documents in Slavic languages, their
normalization/lemmatization, and cross-language matching.  The task
initially covers seven languages and focuses on recognition of four types
of named entities (NEs): person, location, organization, and
miscellaneous, where the last category covers mentions of all other types
of named entities, e.g., event or product.  The input text collection
consists of sets of documents in {seven Slavic languages} collected from
the Web, each collection revolving around a certain entity. The main
rationale of such a set-up is to foster development of ``all-rounder''
named-entity recognition and cross-language entity matching solutions
that are not tailored to specific domains.  The shared task was organised
in the context of the 6th Balto-Slavic Natural Language Processing
Workshop co-located with the EACL 2017 conference.

Several related shared tasks have been organized in the past.  The first
{\em non-English} monolingual NER evaluations---covering Chinese,
Japanese, Spanish, and Arabic---were carried out in the context of the
Message Understanding Conferences (MUSs)~\cite{chinchor:98} and the ACE
Programme~\cite{conf/lrec/DoddingtonMPRSW04}.  The first shared task
focusing on \emph{multilingual} named entity recognition, which covered
some European languages, including Spanish, German, and Dutch, was
organized in the context of CoNLL conferences in
~2002~\cite{TjongKimSang:2002:ICS:1118853.1118877} and in
2003~\cite{TjongKimSang:2003:ICS:1119176.1119195}.  The NE types covered
in these campaigns were similar to the NE types covered in our challenge.
Also related to our task is the Entity Discovery and Linking (EDL)
track~\cite{ji:ea:2014,ji:ea:2015}, which has been part of the NIST Text
Analysis Conferences (TAC).  EDL aimed to extract entity mentions from a
source collection of textual documents in multiple languages (English,
Chinese, and Spanish), and to partition the entities into cross-document
equivalence classes, by either linking mentions to a knowledge base or
directly clustering them.  The fundamental difference between EDL and our
task is that we do not include the sub-task of linking entities to a
knowledge base.

Another task related to cross-lingual NE recognition is NE
transliteration, i.e., cross-lingual linking of NEs across languages
using different scripts.  A series of NE Transliteration Shared Tasks
were organized as a part of NEWS (Named Entity
Workshops)~\cite{duan2016report}, focusing mostly on the Indian and Asian
languages.  In 2010, the NEWS Workshop included a shared task on
Transliteration Mining~\cite{kumaran2010report}, i.e., mining of names
from parallel corpora.  This task included corpora in English, Chinese,
Tamil, Russian, and Arabic.

\todo{add RW for other languages: Ukrainian, Slovak?}Prior work targeting NEs
specifically for Slavic languages includes tools for NE recognition for
Croatian \cite{karan2013croner,ljubesic2013combining}, a tool specifically tailored for NE
recognition in Croatian tweets \cite{baksa2017tagging}, a manually annotated NE
corpus for Croatian \cite{agic2014setimes}, a tool for NE recognition for
Slovene \cite{ljubesic2013combining}, corpus of 11,000 manually annotated NEs
built for Czech~\cite{vsevvcikova2007named}, tools and resources for
fine-grained annotation of NEs in the National Corpus of
Polish~\cite{was:etal:10,sav:pis:11} and a recent shared task on Russian NE
Recognition~\cite{alexeeva2016factrueval}.

To the best of our knowledge, the shared task described in this paper is
the first attempt on multilingual name recognition, normalization, and
cross-language entity matching that covers a large number of Slavic
languages.

This paper is organised as follows. Section~\ref{sec:task} describes the
shared task, while Section~\ref{sec:annotation} describes the annotation
of the dataset.  The evaluation methodology is introduced in
Section~\ref{sec:evaluation}.  The description of participant systems is
provided in Section~\ref{sec:participants} and the results obtained by
these systems are presented in Section~\ref{sec:results}.  Finally,
lessons learnt and main conclusions are discussed in
Section~\ref{sec:conclusions}.

\section{Task Description}
\label{sec:task}


The data for the shared task consists of a collection of text documents
{in seven Slavic languages: Croatian, Czech, Polish, Russian, Slovak,
  Slovene, and Ukrainian. The documents focus around a certain
  entity---e.g., a person or an organization. The documents were obtained
  from the Web, by posing a query to a search engine and parsing the HTML
  of returned documents.}

The task {was} to recognize, classify, and ``normalize'' all named-entity
mentions in each of the documents, and to link together across languages
all named mentions referring to the same real-world entity.

Formally, the Multilingual Named Entity Recognition task includes three
sub-tasks:

\begin{itemize}

\item \textbf{Named Entity Mention Detection and Classification}:
  Recognizing all unique named mentions of entities of four types:
  persons (\textsc{Per}), organizations (\textsc{Org}), locations (\textsc{Loc}), miscellaneous
  (\textsc{Misc}), the last covering mentions of all other types of named
  entities, e.g., products, events, etc.;

\item \textbf{Name Normalization}: {Mapping each named mention of
    an entity to its corresponding base form. By ``base form'' we
    generally mean the lemma (the dictionary form) of the inflected
    word-form, though in some cases normalization should go beyond
    inflection and transform a derived word into a base word's lemma,
    e.g., in case of person possessives (see below).  Multi-word names
    should be normalized to the canonical multi-word expression, rather
    than a sequence of lemmas making up the multi-word expression.}

\item \textbf{Entity Matching}: Assigning to each detected named mention
  of an entity an identifier in such a way that detected mentions of
  entities referring to the same real-world entity should be assigned the
  same identifier, which is referred to as cross-lingual ID.

\end{itemize}

\noindent The task does not include returning positional information of name
entity mentions. Consequently, for all occurrences of the same form of a
mention (e.g., inflected variant, acronym, or abbreviation) of a named entity
within the same document not more than one annotation should be returned,
unless the different occurrences thereof have different entity types (different
readings) assigned to them. Furthermore, distinguishing case information is not
relevant since the evaluation is case-insensitive. In particular, if the text
includes lowercase, uppercase and/or mixed-letter named mention variants of the
same entity, the system response should include only one annotation for all of
these mentions.  For instance, for ``\textit{ISIS}'', ``\textit{isis}'', and
``\textit{Isis}'' (provided that they refer to the same named-entity type),
only one annotation should be returned (e.g., ``\textit{Isis}''). Note that the
recognition of nominal or pronominal mentions of entities is not part of the
task. 

\subsection{Named Entity Classes}

{The task covers four NE classes, defined as follows.}

\begin{description}

\item[Person names (\textsc{Per}).] {This class includes names of
    real persons (and fictional characters).} Person names should not
  include titles, honorifics, and functions/positions. For example, in
  the text fragment ``\textit{CEO Dr. Jan Kowalski}'', only ``\textit{Jan
    Kowalski"}'' should be recognized as a person name. However, initials
  and pseudonyms are considered named mentions of person names and should
  be recognized.  Similarly, named references to groups of people (that
  do not have a formal organization unifying them) should also be
  recognized, e.g., ``\textit{Ukrainians}''. In this context, mentions of
  a single member belonging to such groups, e.g.  ``\textit{Ukrainian}'',
  should be assigned the same cross-language ID as plural mentions, i.e.,
  ``\textit{Ukrainians}'' and ``\textit{Ukrainian}'' when referring to
  the nation should be assigned the same cross-language ID.

  Furthemore, personal possessives derived from a named mention of a person should
  be classified as a person, and the base form of the corresponding
  person name should be extracted. For instance, for
  ``\textit{Trumpov tweet}'' (Croatian) it is expected to recognize
  ``\textit{Trumpov}'', classified with PER and extract the base
  form ``\textit{Trump}''.


\item[Locations (\textsc{Loc}).]

This category includes all toponyms and geopolitical entities (e.g., cities,
counties, provinces, countries, regions, bodies of water, land formations,
etc.) and named mentions of facilities (e.g., stadiums, parks, museums,
theaters, hotels, hospitals, transportation hubs, churches, railroads, bridges,
and other similar urban and non-urban facilities).

Even in case of named mentions of facilities that refer to an organization, the
\textsc{Loc} tag should be used. For example, from the text phrase
``\textit{The Schipol airport has acquired new electronic gates}'' the mention
``\textit{The Schipol airport}'' should be extracted and classified as
\textsc{Loc}.

\item[Organizations (\textsc{Org}).]

This category covers all kind of organizations such as political parties,
public institutions, international organizations, companies, religious
organizations, sport organizations, education and research institutions, etc.

Organization designators and potential mentions of the seat of the organization
are considered to be part of the organization name. For instance, from the text
fragment ``\textit{Citi Handlowy w Poznaniu}'' (a bank in Poznań), the full
phrase ``\textit{Citi Handlowy w Poznaniu}'' should be extracted.
			
When a company name is used to refer to a service (e.g., ``\textit{na
Twiterze}'' (Polish for ``on Twitter'') the mention of ``\textit{Twitter}'' is
considered to refer to a service/product and should be tagged as \textsc{Misc}.
However, when a company name is referring to a service which expresses the
opinion of the company, e.g., ``\textit{Fox News}'', it should be tagged as
\textsc{Org}.			

\item[Miscellaneous (\textsc{Misc}).]

This category covers all other named mentions of entities, e.g., product names
(e.g., ``\textit{Motorola Moto X}''), events (conferences, concerts, natural
disasters, holidays, e.g., ``\textit{Święta Bożego Narodzenia}'' (Polish for ``Christmas''), etc.).

This category does not include the recognition of temporal and numerical
expressions, as well as recognition of identifiers such as email addresses,
URLs, postal addresses, etc.

\end{description}

\subsection{Complex and Ambiguous Entities}

In case of complex named entities, consisting of nested named entities, only
the top-most entity should be recognized. For example, from the text fragment
``\textit{George Washington University}'' one should not extract
``\textit{George Washington}'', but the entire name, namely, ``\textit{George
Washington University}.''

In case a same form (e.g., ``\textit{Washington}'') in two different contexts
in the same document is used to refer to two different real-world entities
(e.g., a person and a location), the system should return two annotations
respectively, where each of them associates the mention with a different
cross-language ID resp.


\subsection{System Input and Response}
\label{sec:protocol}

\paragraph{Input Document Format.}
\label{subsec:input}
%
Documents in collection are represented in the following
format, where the first five lines contain metadata, whereas the core text
to be processed is available from the 6th line till the end of the file.

\begin{small}
\begin{verbatim}
<DOCUMENT-ID>
<LANGUAGE>
<CREATION-DATE>
<URL>
<TITLE>
<TEXT>
\end{verbatim}
\end{small}

\noindent The \verb+<URL>+ field was used to store information on the origin from which 
the text document was created. The values of the metadata fields were computed automatically 
(cf.~Section~\ref{sec:annotation} for details). In particular, the values of the \verb+<CREATION-DATE>+ 
and \verb+<TITLE>+ fields were not provided for all documents either due to unavailability 
of such data or due to web page parsing errors during the creation process. 

\paragraph{System Response.}
\label{subsec:input}
%
For each input document, the systems should return a corresponding file in the following format.
The first line should contain only the \verb+<DOCUMENT-ID>+ field that corresponds to the input file,
whereas each subsequent line should be of the format:

\begin{small}
\begin{verbatim}
<MENTION> TAB <BASE> TAB <CAT> TAB <ID>
\end{verbatim}
\end{small}

\noindent The value of the \verb+<MENTION>+ field should contain the
named-entity mention as it appears in text. The value of the \verb+<BASE>+
field should contain the base form of the entity. Finally, the \verb+<CAT>+ and
\verb+<ID>+ fields are used to store information on the category of the entity
(\textsc{Org}, \textsc{Per}, \textsc{Loc}, or \textsc{Misc}) and cross-lingual
identifier resp.  The cross-lingual identifiers may consist of an arbitrary
sequence of alphanumeric characters.  An example of a system response (for a
document in Polish) is given below.

\begin{small}
\begin{verbatim}
16
Podlascy Czeczeni Podlascy Czeczeni PER 1
ISIS ISIS ORG 2
Rosji Rosja LOC 3
Polsce Polska LOC 4
Warszawie Warszawa LOC 5
Rosja Rosja LOC 3
Magazynu Kuriera Porannego \
   Magazyn Kuriera Porannego ORG 6
\end{verbatim}	    
\end{small}

\section{Data}
\label{sec:annotation}

\subsection{Trial Datasets}

The registered participants were provided two trial datasets:
(1) a dataset related to {\em Beata Szydło}, the current prime minister of Poland, and (2)
a dataset related to ISIS, the so-called ``Islamic State of Iraq and Syria'' terror group.
These datasets consisted of 187 and 186 documents respectively, with
equal distribution of documents across the seven languages of interest.

\subsection{Test Datasets}

Two datasets were prepared for evaluation, each consisting of documents
extracted from the web and revolving around a given entity.  Specifically, the
first dataset contains documents related to Donald Trump, the recently elected
President of United States (henceforth referred to as {\sc Trump}), while the
second dataset contains documents related to the European Commission
(henceforth referred to as {\sc ECommission}).

The test datasets were created in the following manner.  For each entity
of interest, we have posed a separate search query to Google, in each of
the seven target languages.  The query was configured to return links to
documents only in the language of interest.  Next, we extracted the first
100 links (or fewer, in case the search engine did not return 100
links) returned by the search engine, removed duplicate links, and
downloaded the corresponding HTML pages---mainly news articles or
fragments thereof---and converted them into pure text, using a hybrid
HTML parser.  This process was done semi-automatically using the tool
described in~\cite{Crawley:ea:2010}.  In particular, some of the metadata
fields---i.e., creation date, title, URL---were automatically computed
using this tool.

The HTML parsing resulted in extracting texts that may include not only
the core text of a Web page, but also some additional pieces of text,
e.g., a list of labels from a menu, user comments, etc., which might not
necessarily constitute well-formed utterances in the target language.
This occurred in a small fraction of texts processed.  Some of these
texts were included in the test dataset in order to maintain the flavour
of ``real-data.''  However, obvious HTML parser failure, e.g., extraction
of JavaScript code, extraction of empty texts, etc., were removed from
the data sets.  Some of the downloaded documents were additionally
polished by removing erroneously extracted boilerplate content.  The
resulting set of partially ``cleaned'' documents were used to select for
each language and topic circa 20-25 documents for the preparation of the
final test datasets.  Annotation for Croatian, Czech, Polish, Russian and
Slovene were done by native speakers, whereas annotation for Slovak were made 
by native speakers of Czech, capable of understanding Slovak. The annotations 
for Ukrainian were made partially by a native speaker and a near-native speaker 
of Ukrainian. The cross-language alignment of the entity identifiers was done 
by two annotators.

Table~\ref{tab:datasets} provides a more detailed quantitative data about the annotated datasets. A breakdown 
of the entity mention figures with respect to entity class is given in Table~\ref{tab:datasets-2}. 

It is worthwhile mentioning that, for the sake of compliance with the
named entity recognition guidelines described in~Section \ref{sec:task},
documents that included hard-to-decide entity mention annotations were
excluded from the test datasets. {A case in point is a document in
  Croatian that contained the phrase ``\textit{Zagrebačka,
    Sisačko-Moslavačka i Karlovačka županija}''---a contracted version of
  three named entities (``\textit{Zagrebačka županija}'',
  ``\textit{Sisačko-Moslavačka županija}'', and ``\textit{Karlovačka
    županija}'') expressed using a head noun with three coordinate
  modifiers.}

\begin{table}
  \begin{center}
    \begin{footnotesize}
      % \begin{tabular}{|l|c|c|c|c|c|c|}
      \begin{tabular}{lcccc}
        \toprule 
        & \multicolumn{2}{c}{\textbf{{\sc Trump}}} & \multicolumn{2}{c}{\textbf{{\sc ECommission}}} \\
        \cmidrule(lr){2-3}
        \cmidrule(lr){4-5}
        Language &  \#\,docs & \#\,ment & \#\,docs & \#\,ment \\
        \midrule
        Croatian & 25 & 525 & 25 & 436 \\
        Czech & 25 & 479  & 25 & 417 \\
        Polish & 25 & 692  & 24 & 466 \\
        Russian & 26 & 331  & 24 & 385 \\
        Slovak  & 24 & 453  & 25 & 374 \\
        Slovene & 24 & 474  & 26 & 434 \\
        Ukrainian & 28 & 337  & 54 & 1078 \\
        \midrule
        Total & 177 & 3291  & 203 & 3588 \\

        \bottomrule
      \end{tabular}
    \end{footnotesize}
  \end{center}
  \caption{Quantitative data about the test datasets. \#docs and \#ment refer to the number of documents and named-entity mention annotations.}
  \label{tab:datasets}
\end{table}

%%%%%%%%%%%%%%%%%%%%%%%%%%%%%%%%%%%%%%%%%%%%%%%%%%%%%%%%%%%%%%%%%%%%%%%%%%%%%%
\begin{table}
  \begin{center}
    \begin{footnotesize}
      % \begin{tabular}{|l|c|c|c|c|c|c|}
      \begin{tabular}{lcc}
        \toprule 
        Entity type & {\textbf{{\sc Trump}}} & {\textbf{{\sc ECommission}}} \\
        \midrule
        ORG & 18.3\% & 48.4\% \\
        LOC & 26.9\% & 29.1\% \\
        PER & 48.4\% & 11.9\% \\
        MISC & 6.4\% & 9.6\% \\
        \bottomrule
      \end{tabular}
    \end{footnotesize}
  \end{center}
  \caption{Breakdown of the annotations according to the entity type.}
  \label{tab:datasets-2}
\end{table}





\begin{table*}[t]
  \centering
  \setlength\tabcolsep{5pt}
  %\caption{Global caption}
  \begin{minipage}{\linewidth}
    \centering
    %\tablewidth=\textwidth
\scalebox{0.87}{
    \begin{tabular}{| l | l ||l r|l r|l r|l r|l r|l r|l r| }

\hline
{\em Scenario}      & {\sc Trump}     & \multicolumn{14}{c|}{\em Language}                                                                    \\
\hline
Phase               & Metric          & cz       &      & hr   &      & pl   &      & ru   &      & sk   &      & sl   &      & ua   &        \\
\hline
\multirow{6}{*}
{Recognition}       & Relaxed Partial & jhu      & 46.2 & jhu  & 52.4 & pw   & 66.6 & jhu  & 46.3 & jhu  & 46.8 & jhu  & 47.3 & jhu  & 38.8   \\
                    &                 &          &      &      &      & jhu  & 44.8 &      &      &      &      &      &      &      &        \\
\cline{2-16}
                    & Relaxed Exact   & jhu      & 46.1 & jhu  & 50.8 & pw   & 66.1 & jhu  & 43.1 & jhu  & 46.2 & jhu  & 46.0 & jhu  & 37.3   \\
                    &                 &          &      &      &      & jhu  & 43.4 &      &      &      &      &      &      &      &        \\
\cline{2-16}
                    & Strict          & jhu      & 46.1 & jhu  & 50.4 & pw   & 66.6 & jhu  & 41.8 & jhu  & 47.0 & jhu  & 46.2 & jhu  & 33.2   \\
                    &                 &          &      &      &      & jhu  & 41.0 &      &      &      &      &      &      &      &        \\
\hline
Normalization       &                 &          &      &      &      & pw   & 60.5 &      &      &      &      &      &      &      &        \\
\hline
\multirow{3}{*}
{Entity matching}       & Document level  & jhu      &  5.4 & jhu  &  7.3 & jhu  &  6.3 & jhu  & 11.2 & jhu  & 10.1 & jhu  &  9.5 & jhu  &  0.0   \\
\cline{2-16}
                    & Single language & jhu      & 19.3 & jhu  & 17.6 & jhu  & 18.2 & jhu  & 18.9 & jhu  & 22.6 & jhu  & 28.7 & jhu  & 10.7   \\
\cline{2-16}
                    & Cross-lingual   & jhu      &  9.0 &  \multicolumn{12}{c|}{}                                                             \\
\hline
\hline
{\em Scenario}      & {\sc ECommission}        & \multicolumn{14}{c|}{\em Language}                                                                    \\
\hline
Phase               & Metric          & cz       &      & hr   &      & pl   &      & ru   &      & sk   &      & sl   &      & ua   &        \\
\hline
\multirow{6}{*}
{Recognition}       & Relaxed Partial & jhu      & 47.6 & jhu  & 45.9 & pw   & 61.8 & jhu  & 46.0 & jhu  & 49.1 & jhu  & 47.9 & jhu  & 18.4   \\
                    &                 &          &      &      &      & jhu  & 47.3 &      &      &      &      &      &      &      &        \\
\cline{2-16}
                    & Relaxed Exact   & jhu      & 44.4 & jhu  & 43.1 & pw   & 60.9 & jhu  & 44.1 & jhu  & 46.4 & jhu  & 43.9 & jhu  & 14.7   \\
                    &                 &          &      &      &      & jhu  & 42.4 &      &      &      &      &      &      &      &        \\
\cline{2-16}
                    & Strict          & jhu      & 47.2 & jhu  & 46.2 & pw   & 61.1 & jhu  & 46.5 & jhu  & 46.1 & jhu  & 47.8 & jhu  & 10.8   \\
                    &                 &          &      &      &      & jhu  & 44.8 &      &      &      &      &      &      &      &        \\
\hline
Normalization       &                 &          &      &      &      & pw   & 48.3 &      &      &      &      &      &      &      &        \\
\hline
\multirow{3}{*}
{Entity Matching}       & Document level  & jhu      & 25.0 & jhu  & 16.0 & jhu  & 13.7 & jhu  & 13.7 & jhu  & 13.1 & jhu  & 36.8 & jhu  &  0.6   \\
\cline{2-16}
                    & Single language & jhu      & 27.3 & jhu  & 22.1 & jhu  & 17.5 & jhu  & 24.9 & jhu  & 30.6 & jhu  & 32.2 & jhu  &  4.8   \\
\cline{2-16}
                    & Cross-lingual   & jhu      &  2.6 &  \multicolumn{12}{c|}{}                                                             \\
\hline
      \end{tabular}
}
    \caption{Evaluation results across all scenarios and languages.}
    \label{tab:eval-results}

  \end{minipage}%
%  \hfill
\end{table*}



\section{Evaluation Methodology}
\label{sec:evaluation}

Named entity recognition (exact case-insensitive matching) and  Name Normalization (sometimes called “lemmatization”) tasks were evaluated in terms of precision, recall, and F1-scores. In particular, as regards named entity recognition, two types of evaluations were carried out:

\begin{itemize}

\item \textbf{Relaxed evaluation}: An entity mentioned in a given document is considered to be extracted correctly if the system response includes at least one annotation of a named mention of this entity (regardless whether the extracted mention is base form);

\item \textbf{Strict evaluation}: The system response should include exactly one annotation for each unique form of a named mention of an entity that is referred to in a given document, i.e., capturing and listing all variants of an entity is required.

\end{itemize}

\noindent In the relaxed evaluation mode we additionally distinguish between \textit{exact} and \textit{partial matching}, i.e., in the case of the latter an entity mentioned in a given document is considered to be extracted correctly if the system response includes at least one partial match of a named mention of this entity.

In the context of the evaluation we considered various level of granularity, i.e., the performance for: 
(a) all NE types and all languages, (b) each particular NE type and all languages, (c) all NE types for 
each language, and (d) each particular NE type per language. 

As regards the evaluation of document-level and cross-language entity matching task we have adapted the 
Link-Based Entity-Aware metric (LEA)~\cite{DBLP:conf-acl-Moosavi016} which considers how important the entity is and how well it is resolved. LEA is defined as follows. Let $K = \{k_1,k_2,\ldots,k_{|K|} \}$ and $R = \{r_1,r_2,\ldots,r_{|R|} \}$ denote the key entity set and the response entity set respectively, i.e., $k_i \in K$ ($r_i \in R$) stand for set of mentions
of the same entity in the key entity set (response entity set). LEA recall and precision are then defined as follows:

\begin{equation*}
	\mathit{Recall} = \frac{\sum_{k_{i} \in K} (\mathit{imp}(k_i) \times \mathit{res}(k_{i}))}
              {\sum_{k_{z} \in K} imp(k_{z})}
\end{equation*}

\begin{equation*}
	\mathit{Precision} = \frac{\sum_{r_{i} \in R} (\mathit{imp}(r_i) \times \mathit{res}(r_{i}))}
              {\sum_{r_{z} \in R} imp(r_{z})}
\end{equation*}

\noindent where $\mathit{imp}$ and $\mathit{res}$ denote measure of importance and resolution score for an entity respectively.
In our setting we define $\mathit{imp}(e) = \log|e|+1$ for an entity $e$ (in $K$ or $R$), i.e., the more mentions 
an entity has the more important it is. However, in order not to bias the importance of the more frequent entities $\log$ 
is used. The resolution score of key entity $k_i$ is computed as the fraction of correctly resolved co-reference 
links of $k_i$: 

\begin{equation*}
	\mathit{res}(k_i) = \sum_{r_{j} \in R} \frac{\mathit{link}(k_{i} \cap r_{j})}{\mathit{link}(k_{i})}
\end{equation*}

\noindent where $\mathit{link}(e) = (|e| \times (|e|-1))/2$ stands for the number of unique 
co-reference links in $e$. For each $k_i$, LEA checks all response entities to check whether 
they are partial matches for $k_i$. Analogously, the resolution score of response entity $r_i$ is
computed as fraction of co-reference links in $r_i$ that are extracted correctly:

\begin{equation*}
	\mathit{res}(r_i) = \sum_{k_{j} \in K} \frac{\mathit{link}(r_{i} \cap k_{j})}{\mathit{link}(r_{i})}
\end{equation*}

There are various benefits to using LEA.  For example, LEA considers
resolved co-reference relations instead of resolved mentions and has more
discriminative power than other metrics used for evaluation of
co-reference resolution~\cite{DBLP:conf-acl-Moosavi016}.

It is important to stress at this stage that the evaluation was carried
out in ``case-insensitive'' mode: all named mentions in system response
and test corpora were lowercased.


\section{Participant Systems}
\label{sec:participants}
 
Eleven teams from seven countries---Czech Republic, Germany, India,
Poland, Russia, Slovenia and USA---signed up for the evaluation task and
received the trial datasets.  Due to the complexity of the task and
relatively short time available for elaborating a solution, only 2
teams submitted results within the deadline. A total of 2 unique runs 
were submitted.

JHU/APL team attempted only the NER and Entity Matching sub-tasks.  They
employed a statistical tagger called
SVMLattice~\cite{Mayfield:2003:LTU:956863.956921}, with NER labels
inferred by projecting English tags across bitext.  The Illinois
tagger~\cite{Ratinov:2009:DCM:1596374.1596399} was used for English. A
rule-based entity clusterer called "kripke" was used for Entity
Matching~\cite{DBLP:conf/tac/McNameeMFL13}.  
%
The team (code ``{\em jhu}'') attempted all languages available in the
Challenge. More details can be found in~\cite{mayfield:2017}.

{\sc G4.19 Research Group} adapted
Liner2~\cite{series/sci/MarcinczukKJ13}---a generic framework which can
be used to solve various tasks based on sequence labeling, which is
equipped with a set of modules (based on statistical models,
dictionaries, rules and heuristics) which recognize and annotate certain
types of phrases.  The details of tuning Liner2 to tackle the shared task
are described in~\cite{marcinczuk:2017}.
%
The team (code ``{\em pw}'') attempted only the Polish language Challenge.

\comment{
The {\sc DeepNLP} team\footnote{\tt https://github.com/tindzk/bsnlp}
learns entities and their base forms from Wikipedia data
dumps.\footnote{\tt https://dumps.wikimedia.org/} A character-level
recurrent neural network~\cite{DBLP:conf/aaai/KimJSR16} is trained to
predict which characters belong to an entity. A second neural network is
trained on inflected entities and predicts their base forms.
}

Since the announcement of the challenge aroused significant interest of 
the research community the deadline for submitting system responses
has been extended accordingly. Although the aforementioned deadline
goes beyond the time of the publication of this paper, please refer
to the shared task web site~\footnote{http://bsnlp.cs.helsinki.fi/shared\_task.html}
for information on the current list of systems tested and their respective
performance. 

\section{Evaluation Results}
\label{sec:results}

The overall results of the runs submitted as of this writing are presented in
Table~\ref{tab:eval-results}. The figures provided for the recognition are
micro-average F1 scores. As regards normalization the scores provided are
accuracies computed for well-recognized entity mentions for which the system
returned a base form. Finally, as for entity matching the F1 micro-average scores
are provided which were computed using LEA precision and recall values (see Section~\ref{sec:evaluation}).


System {\em pw} performed substantially better on Polish than system {\em
  jhu}.

Considering the entity types, performance was overall better for LOC and
PER, and substantially lower for ORG and MISC, which is not unexpected.

Considering the tested languages and scenarios, system {\em jhu} achieved
best performance on {\sc Trump} in Croatian, its poorest performance was on {\sc ECommission}
in Ukrainian.  System {\em pw} performed better on the {\sc Trump}
scenario than on {\sc ECommission}.  Overall, the {\sc Trump} scenario appears easier
due to the mix of named entities that predominate in the texts.  The {\sc ECommission} documents 
discuss more complex organizations with complex geo-political inter-relationships and affiliations.

Furthermore, cross-lingual co-reference seems to be a difficult task.


\section{Conclusions}
\label{sec:conclusions}


This paper reported on the first challenge on multilingual named entity recognition that aimed at recognizing 
mentions of named entities in web documents in Slavic languages, their normalization/lemmatization, and 
cross-language matching. Although the challenge arouse lot of interest among researchers working 
in this field only two teams submitted results on time, which is most likely due to the complexity
of the task(s) and short time available for elaborating a solution. While drawing far reaching
conclusions from the evaluation of two systems is not possible, we can observe though that 
the overall performance of the two systems on unknown test datasets revolving around a specific entity
is significantly lower than in the case of processing less-morphologically rich languages. 

For the benefit of research on NER-related tasks, including cross-lingual entity matching, for Slavic
languages the challenge was extended going beyond the date of the publication of this paper. The current 
list of systems that has been evaluated on the different task(s) is listed on the shared task web page. 

The test datasets, the corresponding annotations and various scripts used for the evaluation purposes 
are available on the shared task web page.

It is envisaged to extend the challenge through provision of additional test datasets in future, each revolving
around a specific entity, in order to further boost research on developing ``all-rounder'' NER solutions for 
processing real-world texts in Slavic languages and carrying out cross-language entity matching. Furthermore, 
we also intend to extend the set of the languages covered through inclusion of the other Slavic languages 
depending on availability of human annotators. Finally, some work will focus on the refining the NE annotation 
guidelines in order to properly deal with particular phenomena, e.g.,  coordinated NEs and contracted versions of multiple
NEs, which were excluded from the current test datasets.


\section*{Acknowledgments}


We thank Katja Zupan (Department of Knowledge Technologies, Jožef Stefan
Institute, Slovenia), Anastasia Stepanova (State University of New York,
Buffalo), Domagoj Alagić (TakeLab, University of Zagreb), and Olga
Kanishcheva, Kateryna Klymenkova, Ekaterina Yurieva (the National
Technical University, Kharkiv Polytechnic Institute), who contributed to
the preparation of the Slovenian, Russian, Croatian, and Ukrainian test
data.  
%
We are also grateful to Tomaž Erjavec from the Department of Knowledge
Technologies, Jožef Stefan Institute in Slovenia, who contributed various
ideas.
%
Work on Czech and Slovak was supported by Project MediaGist, EU's FP7
People Programme (Marie Curie Action), no. 630786.

The effort of organizing the shared task was supported by the Europe Media Monitoring (EMM) 
Project carried out by the Text and Data Mining Unit of the Joint Research Centre of the 
European Commission.

\bibliographystyle{eacl2017}
\bibliography{sharedBSNLP2017}




\end{document}
